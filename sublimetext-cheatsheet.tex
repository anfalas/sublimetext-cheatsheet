\makeatletter\def\input@path{{applekeys/}}\makeatother
\PassOptionsToPackage{table}{xcolor}
\documentclass[10pt,a4paper,landscape]{article}
\usepackage{multicol}
\usepackage{calc}
\usepackage{ifthen}
\usepackage[landscape]{geometry}
\usepackage{titlesec}
\usepackage{graphicx}
\usepackage{ stmaryrd }
\usepackage{tikz}
\usetikzlibrary{shadows, calc}
%\usepackage[table]{xcolor}
%-----------------------------------------------------------------------

% Sublime Text 2 cheat sheet

% This sheet is meant to be compiled with pdflatex.

% This code is based on the LaTeX cheat sheet found here :
% http://www.stdout.org/~winston/latex/
% by Winston Chang

% Licensed under Creative Commons Attribution-NonCommercial-ShareAlike 3.0 Unported License.

%-----------------------------------------------------------------------
% Change default fonts for the document
%
\usepackage[utf8]{inputenc}
\usepackage[T1]{fontenc}
\usepackage[scaled]{helvet}
\renewcommand*\familydefault{\sfdefault}
\renewcommand*\ttdefault{txtt}
%-----------------------------------------------------------------------

% Setting the margins at 1cm everywhere
\geometry{top=1cm,left=1cm,right=1cm,bottom=1cm} 

% Turn off header and footer
\pagestyle{empty}
 
\makeatletter

\newcommand{\vcenteredinclude}[2]{\begingroup
\setbox0=\hbox{\includegraphics[#2]{#1}}%
\parbox{\wd0}{\box0}\endgroup}

\newcommand*\keystroke[1]{%
  \tikz[baseline=(key.base)]
    \node[%
      draw,
      fill=white,
      drop shadow={shadow xshift=0.25ex,shadow yshift=-0.25ex,fill=black,opacity=0.75},
      rectangle,
      rounded corners=2pt,
      inner sep=1pt,
      line width=0.5pt,
      font=\scriptsize\sffamily
    ](key) {~#1~\strut}
  ;
}

% Don't print section numbers
\setcounter{secnumdepth}{0}


\setlength{\parindent}{0pt}
\setlength{\parskip}{0pt plus 0.5ex}

\renewcommand{\familydefault}{\sfdefault}

\titleformat{\section}[block]%              
    {\tikz[overlay] \shade[left color=gray!10,right color=gray] (0,-1ex) rectangle (\linewidth,1.5em); \Large}%    
    {\thesection}%                   
    {1em}%
    {}

\titleformat{\subsection}[block]%              
    {\large}%    
    {\thesubsection}%                   
    {1em}%
    {}
% -----------------------------------------------------------------------

%-----------------------------------------------------------------------
% Defines a command to easily insert keyboard modifier from pdf images
%
\newcommand{\key}[1]{
  \scalebox{0.012}{\includegraphics{{#1}.pdf}}
}
%-----------------------------------------------------------------------

\begin{document}

%\renewcommand{\familydefault}{\sfdefault}

\newcommand{\ret}{\keystroke{$\hookleftarrow$}}
\newcommand{\shift}{\keystroke{$\Uparrow~$}}
\newcommand{\alt}{\keystroke{\key{option}}}
\newcommand{\ctrl}{\keystroke{\key{control}}}
\newcommand{\cmd}{\keystroke{\key{command}}}
\newcommand{\up}{\keystroke{$\uparrow$}}
\newcommand{\down}{\keystroke{$\downarrow$}}
\newcommand{\leftarr}{\keystroke{$\leftarrow$}}
\newcommand{\rightarr}{\keystroke{$\rightarrow$}}
\newcommand{\bkspc}{\keystroke{$\longmapsfrom$}}
%\newcommand{\ctrl}[1]{\texttt{\keystroke{Ctrl}#1}}
%\newcommand{\ctrl}[1]{\texttt{\keystroke{\key{control}}#1}}

\raggedright
\footnotesize
\begin{multicols}{3}

\begin{center}
     \vcenteredinclude{sublime_text.png}{width=32pt}\Large{\textbf{ublime Text 2 Cheat Sheet for Mac}} \\
\end{center}

\rowcolors{1}{gray!20}{white}

All the commands of Sublime Text 2 can be accessed from the command palette.

\section{General}
\begin{tabular}{p{3cm}p{\linewidth - 3.9cm}}
\cmd{\shift\keystroke{P}} & \textbf{Command palette} \\
\ctrl{\keystroke{`}}  & Console \\
\cmd{\keystroke{,}} & (User) Preferences \\
\cmd{\keystroke{F}} & Full Screen Mode \\
\cmd{\shift\keystroke{F}} & Distraction-Free Mode \\
\cmd{\keystroke{K} - \keystroke{B}}  & Toggle side bar \\
\cmd{\keystroke{B}}  & Build \\
\end{tabular}

\section{File Navigation}
\begin{tabular}{p{3cm}p{\linewidth - 3.9cm}}
\cmd{\keystroke{P}}  & Go to anything \\
\cmd{\keystroke{T}}  & Go to file \\
\cmd{\alt \rightarr}  & Next file \\
\cmd{\alt \leftarr}  & Previous file \\
\cmd{\keystroke{R}}  & Go to symbol/method (\verb|@|) \\
\cmd{\keystroke{;}}  & Go to word  (\verb|#|)\\
\ctrl{\keystroke{G}}  & Go to line number (\verb|:|) \\
\cmd{\shift\keystroke{N}} & New window/project \\
\end{tabular}

\section{Code Navigation}
\begin{tabular}{p{3cm}p{\linewidth - 3.9cm}}
\cmd{\leftarr}  & End of line \\
\cmd{\rightarr}  & Beginning of line \\
\cmd{\up}  & Top of file \\
\cmd{\down}  & Bottom of file \\
\alt{\rightarr}  & Next word \\
\alt{\leftarr}  & Previous word \\
\end{tabular}

\section{Splits/Tabs}
\begin{tabular}{p{3cm}p{\linewidth - 3.9cm}}
\cmd{\alt \keystroke{1}} & Single Column\\
\cmd{\alt \keystroke{[2,3,4]}} & 2,3,4 Columns\\
\cmd{\alt \keystroke{5}} & Grid \\
\ctrl{\keystroke{[1,2,3,4]}} & Focus group \\
\ctrl{\shift \keystroke{[1,2,3,4]}} & Move file to group \\
\cmd{\keystroke{[1,2,3...]}} & Select tab \\
\end{tabular}

\section{Selection}
\begin{tabular}{p{3cm}p{\linewidth - 3.9cm}}
\cmd{\keystroke{L}} & Select line (repeat select next lines) \\
\ctrl{\shift \keystroke{M}} & Select all between brackets \\
\cmd{\keystroke{D}} & Select word (repeat select others occurrences in context for multiple editing)\\
\cmd{\ctrl \keystroke{G}} & Select all occourencies of word \\
\ctrl{\shift \up} & Select column upwards \\
\ctrl{\shift \down} & Select column downwards \\
\alt (mouse-drag) & Select column \\
\alt \shift \up & Insert selection cursor one line up \\
\alt \shift \down & Insert selection cursor one line down \\
\end{tabular}

\section{Code Editing}
\begin{tabular}{p{3cm}p{\linewidth - 3.9cm}}
\cmd{\keystroke{X}} & Delete line \\
\cmd{\ret} & Insert line after \\
\cmd{\shift \ret} & Insert line before \\
\cmd{\shift \keystroke{D}} & Duplicate line(s) \\
\cmd{\keystroke{J}} & Join line(s) \\
\cmd{\keystroke{/}} & Toggle comment for current line/selection \\
\cmd{\alt \keystroke{/}} & Toogle block comment \\
\ctrl{\keystroke{M}} & Jump/switch between matching brackets \\
\ctrl{\shift \keystroke{K}} & Delete line(s) \\
\cmd{\keystroke{K} - \keystroke{K}} & Delete from cursor to end of line \\
\cmd{\keystroke{K} - \bkspc} & delete from cursor to start of line \\
\cmd{\keystroke{K} - \keystroke{U}}  & Uppercase currend word or selection \\
\cmd{\keystroke{K} - \keystroke{L}}  & Lowercase currend word or selection \\
\ctrl{\shift \up}  & Move line/selection up \\
\ctrl{\shift \down}  & Move line/selection down \\
\cmd{\keystroke{]}} & Indent current line(s) \\
\cmd{\keystroke{[}} & Unindent current line(s) \\
\cmd{\shift \keystroke{V}} & Paste and indent correctly \\
\cmd{\keystroke{Y}} & Redo or repeat \\
\ctrl{\keystroke{~~Space~~}} & Autocomplete (repeat to select next suggestion) \\
\cmd{\keystroke{U}} & Soft undo (movement undo) \\
\cmd{\shift \keystroke{U}} & Soft redo (movement redo) \\
\end{tabular}

\section{Find/Replace}
\begin{tabular}{p{3cm}p{\linewidth - 3.9cm}}
\cmd{\keystroke{F}} & Find \\
\cmd{\alt \keystroke{F}} & Find and replace \\
\cmd{\keystroke{G}} & Find next \\
\cmd{\shift \keystroke{G}} & Find previous \\
\cmd{\alt \keystroke{G}} & Find next occurrence of current selected word \\
\cmd{\ctrl \keystroke{G}} & Select all occurrences of current word for multiple editing \\
\cmd{\keystroke{I}} & Incremental find \\
\cmd{\shift \keystroke{F}} & Find in files \\
\end{tabular}

\section{Bookmarks}
\begin{tabular}{p{3cm}p{\linewidth - 3.9cm}}
\cmd{\keystroke{F2}}  & Toggle bookmark \\
\keystroke{F2} & Next bookmark \\
\shift \keystroke{F2} & Previous bookmark \\
\ctrl{\shift \keystroke{F2}} & Clear bookmarks
\end{tabular}

\rule{\linewidth}{0.25pt}
\scriptsize

Copyright \copyright\ 2013 Gregor Longariva

http://blogs.fau.de/faumac




Credits to: http://pragamticstudio.com, http://www.gsmproductions.org/misc/sublime.html and https://gist.github.com/lucasfais/1207002

\end{multicols}
\end{document}
